\section{Standard Module \sectcode{digisign}}
\label{module-digisign}

This module provides a simple class to create keys for generating and
verifying \dfn{digital signatures}.  These digital signatures act like
handwritten signatures in the real world; only one key can sign a
message, but anyone can use the key to verify that the signature
corresponds to the message.  If even a single bit of the message is
changed, verifying the signature will fail.

The algorithm used is Nyberg and Ruppel's q-NEW.  The \code{Key} class
contains a single q-NEW key.  A newly generated key is a \dfn{private
key}.  It includes private information, and can both sign messages, or
verify a previously generated signature for a given message.  It's
possible to create a copy of the key without the private information,
using the \code{publickey} method; this public key can only verify
signatures produced with the private key, and thus doesn't have to be
kept secret.

For example, you could generate a private key, create the matching
public key, and send the public key to all your correspondents.  Then,
when you wish to send a message, you can sign it using the private
key, and send the signature along with the message; your
correspondents can then verify the signature using the public key,
assuring them that the message wasn't changed while being transmitted.


\stmodindex{digisign}
\renewcommand{\indexsubitem}{(in module digisign)}

\subsection{The {\tt Key} class}

The \code{Key} class encapsulates a single key.  To create a new key
object, call either \code{digisign.new()} or \code{digisign.Key()},
which will return an empty \code{Key} instance, and then call the
\code{generate()} method of the empty instance to create a random 
private key.  To obtain a \code{Key} instance containing only the
public key, call the \code{publickey} method.  Both private and public
\code{Key} instances can be pickled.

The \code{generate} and \code{sign} methods require random data.  Both
those functions have a keyword parameter named \code{randfunc}; this
parameter must be a function (or a bound method) which will take a
single integer parameter N and return N bytes of random data.

It is critical that the random number generator be
\dfn{cryptographically strong}; consult Schneier's book for the
details.  You may wish to use the \code{RandomPool} class described
below, which is such a generator.  Be sure you know what you're doing
before writing your own generator, since keys produced with a poor
generator will offer little or no security.  Simple linear generators,
like the one in the
\code{whrandom} module, are not strong, and should never be used with
this module.

\begin{funcdesc}{cansign}{}
Return true if this key contains the private information required to
generate new signatures.  Return false if the key can only be used to
verify signatures.
\end{funcdesc}

\begin{funcdesc}{publickey}{}
Return a new \code{Key} instance containing only the public key
information.  This new instance cannot generate signatures, though it
can verify them.
\end{funcdesc}

\begin{funcdesc}{generate}{bits\, randfunc\, progress_func}
Generate a new private key that can sign messages and verify its own
signatures.  \var{bits} is the size of the key in bits, and has a
minimum value of 384; as this value increases, the key becomes harder
to break, and signing and verifying gets slower.  A value of 768 or
1024 is recommended for security; key sizes of 384 or 512 bits may be
used in testing for the sake of speed, but they're easily breakable in
1996.

\var{randfunc} is a function that generates random strings, as described above.
The function \var{progress_func} will be called at various points
during key generation, and passed a string explaining the algorithm's
progress.  Key generation is a lengthy process, so this can be used
to keep the user informed while it's going on.
\end{funcdesc}

\begin{funcdesc}{sign}{M \optional{\, randfunc\, K} }
Generate and return a string containing a signature for the short
message string \var{M}.  \var{M} is limited in size to 20 bytes or
less.  If you wish to sign a larger piece of data, like a 1K email
message, the usual procedure is to take the hash of the message and
use that as \var{M}.  This can be done using the \code{md5} module.

The optional parameter \var{K} is a random
value required to produce a signature; it can be a string of no more
than 20 bytes in length.  Alternatively (and more easily), the
\var{randfunc} parameter can be provided, and a value of \var{K} will
be automatically generated.

{\bf Important:} The same value of \var{K} must \emph{never} be used
twice, because that would allow a third party to determine the private
key data.  An attacker can save every message and signature produced
by a key, and try all possible pairings in case a value of \var{K} was
reused.  Once an attacker has the private key data, they can sign
messages as they like.  (If automatically generated, \var{K} is a
159-bit number, so the odds of the same number being used twice are
infinitesimal as long as a good random number generator is used.)
\end{funcdesc}

\begin{funcdesc}{verify}{M\, signature}
Return true if \var{signature} matches the message \var{M}.  If they
don't match, return false; this may happen because \var{M} or
\var{signature} was modified, or because the signature was generated
with a different key.
\end{funcdesc}

\subsection{The {\tt RandomPool} class}

The \code{RandomPool} class provides cryptographically strong
generation of random data.  

\begin{funcdesc}{RandomPool}{\optional{seed}}
Return a \code{RandomPool} instance, initialized using the string
\var{seed}.  The instance can then be used to get random data.  
The more randomness that can be in the \var{seed} string, the more
secure the resulting random data will be.

If \var{seed} is omitted, the class will try to get random data on its
own, reading from \file{/dev/urandom} on those systems that support
it, and otherwise using the current system time.  The system time
isn't a very good source of randomness, so a warning will be printed
to \code{sys.stdout}.
\end{funcdesc}

\begin{funcdesc}{random}{N}
Return an \var{N}-byte string containing random data.  

When bound to an instance, this method can be passed as the
\var{randfunc} parameters to \code{Key}'s \code{generate} and
\code{sign} methods.  Simply create a \code{RandomPool} object
\var{R}, and pass \code{\var{R}.random} as the parameter.
\end{funcdesc} 


\begin{seealso}
\seetext{XXX Nyberg and Ruppel's paper},
\seetext{Schneier's book}
\end{seealso}

